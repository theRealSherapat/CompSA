\chapter{Experiments and results}
\label{chap:experiments_and_results}
\besk{Et kapittel der du har satt opp (med hyper-parametere og miljø-variabler f.eks. i en oversiktlig tabell) eksperimenter, kjørt simulation-runs med disse verdiene, og viser hva resultatene ble. Resultater kan være \textbf{performance plot}s (ift. harmonic synch.-times for diverse hyper-parametere og miljøvariabler (samt om robotene er homogene eller heterogene og isåfall hvordan), der plottet må lages spesifikt for hvert eksperiment, og kan f.eks. være boxplot eller tabeller med performance-/hsynchtime-/simulation-time(s)-verdier), \textbf{harmonic synchrony detection plot}s (altså hvordan harmonic synch. ble detectet i et simulation-run), \textbf{phase-\&frequency-plot}s (altså hvilke fase- og frekvens-verdier robotene hadde iløpet av simulation-run'et, via \textit{plot\_PhaseFrequencyPlot\_for\_SimRun.py}), eller \textbf{synchrony-evolution plot}s (der 'towards\_k\_counter'en iløpet av simulation-run'et plottes, via \textit{plot\_SynchronyEvolutionPlot\_for\_SimRun.py})}

\besk{Helst gode eksperimenter som motiveres og forklares, settes opp, og utføres m/resultater man diskuterer og analyserer. Det er bra å evaluere fra flere synspunkt \tcol[gray]{med flere research methods og hvis tid}}

Unless otherwise is stated, the heterogenous visual looks of the Dr. Squiggles robots in Unity have no real difference in the simulator and is only that (for visual looks), not implying other values or methods used.

	\section{Solving the simpler $\phi-$problem}
	This is the section for the experiments attempting to solve the first and simpler problem, namely synchronizing the phases $\phi_i$ of all agents $i$. \nl
	
	\subsection{Reproducing K. Nymoen's phase synchronization}
	In order to see that our developed synchronization simulator in Unity yields more or less the same results as K. Nymoen et al.'s firefly system \cite{nymoen_synch}, similar experiments as reported in their paper are performed. These tell us whether differences in performance, in terms of synchronization times (s), is simply due to implementation differences, or actually because of the used synchronization methods and hyperparameters in question. First off, for the $\phi$ problem of synchronizing only phases (i.e. already having synchronized frequencies), K. Nymoen et al.'s phase adjustment method (presented in Section \ref{nymoen_phase_adjust}) is experimented with for varying phase coupling constants $\alpha$, as also performed and reported in K. Nymoen et al.'s paper \cite{nymoen_synch} (c.f. their Fig. 7):  \nl
	
	\tcol[blue]{FYLL INN}
	
	\begin{figure}[ht!]
		\centering
		% \includegraphics[width=0.7\]{Assets/DocSegments/Chapters/ExperimentsAndResults/Figures/}
		\caption{Synchronization times (s) for 6 robots with initially random and unsynchronized phases but equal and fixed frequencies (1Hz), with varying $\alpha$ values. 30 simulation runs per $\alpha$ value. $\beta=0$, $t_{ref\_percentage\_of\_period} = 5\%$ yields $t_{ref}=0.05*1s=50ms$, since the oscillator period for all robots is $1/fixed\_frequency=1/1Hz=1s$.}
	\end{figure}
	
	
	
	\subsection{Hyperparameter tuning}
	Tuning hyperparameters $\alpha$ and $t_{ref\_percentage\_of\_period}$ for several robot collective sizes, according to the performance scores of how long it takes the robot collective to achieve harmonic synchrony.
	
	\subsection{Comparing phase adjustment methods}
	Now, an experiment follows where we investigate the validity of the claimed [] benefits of performing bi directional phase adjustments—both inhibitory and excitatory—compared to simply adjusting phases in an excitatory way (i.e. only ``pushing'' other ocillators's phase further or higher when firing, never ``holding'' or ``dragging'' it back).
	
	% GAMMEL EPA1-CAPTION: Performance-plot: harmonic synchronization-times from initial simulator-experiment when synchronizing phases $\phi_i$ for all agents $i$, where all phases are initially uniformly randomized between 0 and 1, and eventually synchronize and align. We here measure how long it takes 6 agents to synchronize their phases to each other, given the two different phase-adjustment methods. 30 individual runs per phase-adjustment method were performed in Unity for a collective-size of 6 agents, and $\alpha=0.2$ e.g.
	
	
	
	
	
	\section{Solving the harder $\phi\&\omega-$problem}
	
	This is the section for the experiments attempting to solve the second and harder problem of synchronizing both phases $\phi_i$, as well as frequencies $\omega_i$, for all agents $i$.

	\subsection{Reproducing K. Nymoen's frequency synchronization}
	
	\paragraph{Ordering by phase couplings}
	Given that K. Nymoen et al. \cite{nymoen_synch} do not mention their $\beta$ value in their frequency synchronization experiment where they order for different phase coupling values $\alpha$, an \textit{empirically decent} $\beta$ value of 0.4 is chosen for this experiment. What \textit{empirically decent} refers to in this case are K. Nymoen et al.'s findings in the results of their last experiment where (frequency) synchronization times for various $\beta$ values were evaluated; yielding $\beta=0.4$ to be a good value, with no further improvement in synchronization performance when increased further. Also, when empirically observing the visual and aural signals being transmitted through the synchrony simulator in Unity, this value of $\beta=0.4$ leads to stable simulation runs where the robot collective manages to achieve harmonic synchrony, which is very desirable when testing how fast they can synchronize. If we increase $\beta$ much more in the Unity simulator, we quickly start to observe highly unstable simulation runs, hence calling for special rules like K. Nymoen et al. implemented in their firefly system—which we so far have managed without.
	
	\tcol[blue]{FYLL INN}
	
	As explained earlier, in the this system we rather use $t_{ref\_percentage\_of\_period}$ that depends on each robot's individual oscillator frequency, rather than a fixed $t_{ref}$ for all robots (for stabilization purposes). Hence, our systems cannot for the frequency synchronization problem be said to be completely analogous to K. Nymoen et al.'s firefly system, as the novel Unity simulator uses more dynamical and ever-changing refractory periods than those used by K. Nymoen et al. \cite{nymoen_synch}.
	
	\paragraph{Ordering by frequency couplings}
	
	\subsection{Hyperparameter tuning}
	Tuning hyperparameters $\beta$ and $m$ for several robot collective sizes according to performance scores.