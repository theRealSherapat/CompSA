\chapter{Tools and software}
\label{chap:tools_and_software}


	% --- META INFO START: ---

	\besk{ "which describes what you've used" — ish Kyrre }

	\besk{ Kan flyttes til en egen seksjon hvis dette kapittelet ikke ville vært så stort \tcol[gray]{(jf. 'ThesisChecklist' på Robin-wikien)} }

	\besk{ \tcol[gray]{(Hentet fra Tønnes sin master, om Tools and engineering)} En introduksjon til de forskjellige verktøyene og prosessene brukt iløpet av masteroppgaven. Fokuser på fysisk arbeid gjort, og ingeniør-delene av masteroppgaven, inkludert 3D-design av de fysiske robotene, valg av deler, simulering i systemer, og testingen, valideringen, og verifikasjonsmetoder brukt i oppgaven. Gjerne også en oversikts-tabell av verktøy og programvare brukt }

	% --- META INFO STOP. ---


\section{Software}

\begin{itemize}
	\item Notepad++ v8.1.9.2 (64 bit). For writing my master's thesis and code.
	
	\item C\#.
	
	\item Unity Version 2021.2.0f1. Unity is originally a game-development platform, but can also be used to make \inkl{realistic} simulations containing physical rigid-bodies using the <bla.bla Rigidbody>-physics engine.
	
	\item Python 3.10.0. for plotting and analysing data.
	
	\item Inkscape 1.1.1 for editing and creating vector graphics.
	
	\item Gimp 2.10.30 for editing and creating raster graphics.
	
	\item Audacity 3.1.0 for plotting audible waveforms, as in Figure \ref{fig:waveform}, as well as frequency spectra of such recorded waveforms, as in Figure \ref{fig:frequency_spectrograms}.
	
	\item LMMS 1.2.2, being a music-making system and digital synthesizer, for synthesizing harp-tones used for designing ``fire''-signals.
\end{itemize}

% --- \section END. ---



\section{Hardware}

\begin{itemize}
	\item CXT USB-Microphone for recording the hummed waveform, as in Figure \ref{fig:sub:G3_hummed_waveform}.
\end{itemize}

% --- \section END. ---