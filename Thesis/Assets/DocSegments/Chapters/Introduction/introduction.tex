\chapter{Introduction}
\label{chap:introduction}	

	% --- META INFO START: ---

	\besk{ Illustrative explaining System overview figure: En kjerne-illustrativ figur av det utviklede og endelige resultat-systemet, som i Benedikte sitt PSCA-system }
	
	

	% --- META INFO STOP. ---



\section{Motivation}

In this MSc-thesis, we will explore an exciting and relatively new translation of the concepts and notions regarding self-awareness — as they pertain to humans and animals especially — from the domain of Psychology, into the domain of Computation and Engineering.

We will in this project attempt to implement and explore whether, and indeed in what way (how), a computational system (like computational agents/agent-collectives in a computer-simulation, and even on physical musical robots if time permits it) can exhibit and display Self-Awareness (and corresponding Self-Expressive) capabilities, comparable to how they are perceived in humans and animals.

We will in this thesis project investigate methods and models for computational self-awareness in multi-robot systems, with application to the musical robotics domain. We will investigate whether increased levels of computational self awareness leads to more intelligent systems among other things. We want to investigate whether endowing computational systems, in this case specifically a musical robot system, significantly increases performance in a given task or not. We want to discover whether increased self awareness capibilities will lead to better decision making (or \textit{self expression} \cite{sacs16_ch2}), versatility and ability (in its self expression) to adapt quickly and online in a rapidly dynamic and everchanging enviroment, as they continually learn by themselves further handling continuous environmental change. We generally want to discover the effects of endowing computational systems/robots with self-awareness capabilities.

Designing and predicting all possible scenarios of a computing system at design-time is often hard, and sometimes impossible (in the case of unpredictable faults e.g.). If one wants to achieve coordination and continuous adaptation of a system or of system-components (for example in a collective) -- some sort of intelligence might be necessary to endow it with. Endowing computing systems with Self-Awareness can be beneficial in several respects, including a greater capacity to adapt, to build potential for future adaptation in unknown environments, and to explain their behaviour to humans and other systems. As we can see in various Music Technology Systems, this endowing can also give rise to interesting cooperative and coordinating behaviour.

The problem of this thesis will mainly consist of studying the effects differing Self-Awareness levels, varying collective-sizes, levels of task difficulty (like more complex behaviours, and limited communication) - have on usefulness, system dynamics, overall performance, and scalability (primarily within the domain of Musical Multi-Robot/-Agent collectives).

Engineering a computing system for a certain environment often requires some knowledge of said environment; both on the end of the system-creator, as well as for the computing system in turn. This is at least the case in autonomous computing, where computing systems are supposed to be able to observe, learn, adapt, and act on their own, independently from their creator.

Foreseeing and predicting all possible (future) states and scenarios a computing system will face during its lifetime, in often complex, dynamic, and ever-changing environments, at design-time is often hard, and sometimes impossible (in the case of coincidental faults e.g.). This calls for some online and continuous learning. With sslf awareness, online and continuous learning is achieved to a higher degree in contrast with other approaches (like ODA and MAPE-K), due to the limitations and downsides of these older approaches, as well as the advantages and upsides to considering computational self-awareness in computing systems. If one wants to achieve continuous adaptation of a system or of system-components (e.g. in a collective) — some sort of intelligence might be necessary to endow it with. Endowing computing systems with Self-Awareness can be beneficial in several respects, including but not limited to a greater capacity to adapt, to build potential for future adaptation in unknown environments, and to explain their behaviour to humans and other systems \cite{sacs17_ch3}.

Especially explainability is with time becoming more and more relevant, also within artificial intelligence (AI) (hence the popularity of the term \textit{explainable AI}), as increasingly autonomous and automatically systems are making real life decisions with serious consequences increasingly on a day to day basis. L. A. Dennis and M. Fisher present explaining and verifiable agent-based systems, where rational agents who possess goals, beliefs, desires, intentions e.g. make decisions that in their opinion should be able to be questioned and giving an answer when prompted \cite{verifiable_and_questionable_agents}. Self awareness enables such questioning and answering by autonomous agents, as the agents themselves, since they are themselves aware of their knowledge, thoughts, goals, desires etc., can explain what lead to their actions. Or in other words, questions like ``what type of self awareness \cite{sacs16_ch2} (knowledge) lead to a certain self expression \cite{sacs16_ch2} (action)'' can be answered by self aware and self expressive agents.

Self-awareness concepts from psychology are inspiring new approaches for engineering computing systems which operate in complex dynamic environments \cite{sacs16_ch2}. As we can see in various Music Technology Systems, this endowing can also give rise to interesting cooperative and coordinating behaviour.

Extreme challenges and physical barriers within communication, like too large latencies, bottlenecks presented by using the same limited bandwidths, short possible ranges, or the inability to use global satellite systems e.g. in underwater vehicles \cite{petillot_underwater_robots}, leads to the necessity of enabling systems to autonomously and online control themselves and perform missions without communication from remote operators like humans. Hence, technological development and research has e.g. within underwater robotics gone from enabling remotely operated vehicles (ROVs) in the 1980s (for e.g. oil and gas exploitation at depths unreachable to human divers), over to more self adaptive and autonomous underwater vehicles (AUVs) closer to in the 21st century —the demand of which is expected to grow 37\% in the 2018-2022 period \cite{petillot_underwater_robots}. As a result, the underwater robots can in this instance thus, due to the increased degree of autonomy and consequently implicit reduction in communication need, travel on its own on even longer and extensive missions without the need for a live connection to any human operator—e.g. on seabed mapping missions, underwater machinery or structure maintenance, as well as seabed cleaning.

However, and especially within collectives of multiple robots \cite{cocoro, swarm_bot}, communication between individuals—even though they might be autonomous individual robots—is still a challenge one has to deal with when designing them. Coordination, for humans and robots alike, often tend to demand a lot of communication. To add on top of this, when networks or collectives of sub-systems like wireless sensors are communicating with each other, they often have their own internal clock which often do not align. Hence, we see that both the magnitude of communication needed, as well as the inherent challenges with communication, call for effective approaches to communication with coordination as the goal.

Various attempts at synchronizing messages and communication, in order to e.g. align messages exchanged at unsynchronized intervals, has been made \cite{tungvinte_sync_protocols}; however, such attempts and protocols often require the computation of message exchange and processing, which wastes the limited computation capability of nodes and causes communication delays \cite{konishi_kokame}. To this, e.g. K. Konishi \& H. Kokame \cite{konishi_kokame} points out how for synchronized pulse coupled oscillators (PCOs), such computation is not required. If internal clocks in communicating nodes are instead altered and synchronized to other nodes's clocks—instead of letting all clocks stay unsynchronized and constantly trying to compensate for this— it is rather apparent that getting rid of this need for compensation will considerably reduce computation needs while synchronized communication is still achieved.

Thus, by achieving time synchronization \cite{tungvinte_sync_protocols} through synchronizing pulse coupled oscillators, both the computational load, in addition to the need for communication, is considerably reduced; hence freeing up and saving valuable resources like e.g. processing power and energy (battery) in networks of individuals and possibly autonomous nodes, agents, or robots.




% --- \section END. ---



\section{Goal of the thesis} % a.k.a. research goals.


	% --- META INFO START: ---

	\besk{ "to make the reader better understand what the thesis is about"—Jim, og "en rød tråd?"—Sigmund }

	\besk{ Spesifikke mål \tcol[gray]{(goals/aims)} med master-oppgaven/-prosjektet. Hva jeg vil vise/demonstrere til folk \tcol[gray]{(leserne f.eks.)} om self-awareness \tcol[gray]{(SA)} og hvordan \tcol[gray]{(detaljene av beskrevet i \textbf{Implementation})} — noe jeg håper å enten bekrefte eller avkrefte med eksperimenter og resultater i \textbf{Experiments and results} }

	% --- META INFO STOP. ---


The specific aim of this thesis is to explore and investigate the effects of endowing robot systems with increased self-awareness abilities, and thus leads to the following research questions: \nl




\textbf{Research Question 1}:

Will performance in collective multi-robot systems increase as the level of \textit{self-awareness} increases? Specifically, will increased levels of self-awareness in the individual agents/musical robots lead to the collective of individuals being able to synchronize to each other faster than with lower levels of self-awareness? \nl

\textbf{Research Question 2}:

Will increased levels of self-awareness lead to more robustness and flexibility in terms of handling environmental noise and other uncertainties — specifically in the continued ability of musical robots to synchronize to each other efficiently despite these difficulties/challenges? \nl

% --- \section END. ---



\section{Outline}
En fin \tcol[gray]{(Eagle's-eye)} oversikt over strukturen til hele dette dokumentet fra nå av, og utover.

% --- \section END. ---



\section{Scope and delimitations}


	% --- META INFO START: ---

	\besk{Hva oppgaven forsøker å oppnå eller besvare, og hva den ikke forsøker å oppnå eller besvare}

	% --- META INFO STOP. ---



% --- \section END. ---



\section{Contributions}


	% --- META INFO START: ---

	\besk{De viktigste bidragene av MSc thesis-arbeidet summert \tcol[gray]{(for å få det til å stå frem/ut bedre enn å bare "gjemme" det i den siste delen av oppgaven)}. Fra Mats: HUSK Å SELG!}

	% --- META INFO STOP. ---


This thesis work has led to some main novel contributions. The main contribution is that a novel synchronization-simulator has been created and developed in Unity, where various and customly made synchronization-methods can be implemented in a robot collective (either homogenously but also heterogenously in each robot); as well as that users of the simulator can both see and hear the robots's interactions and entrainment towards synchrony (including whether they manage to finally achieve synchrony, and in that case how long it takes them to get there). This novel framework opens, for everyone interested in it, for the possibility of experimenting with creative and novel synchronization-methods, in order to qualitatively and quantitatively assess their efficiency at achieving synchrony. A reason for this is due to the seamless scalability and dynamism, both in terms of robot-collective size, heterogenous or homogenous agents, as well as how a new update-function (for both phase-updating and frequency-updating) can so easily be added in and coded with a couple of lines.

The developed synchronization simulator in Unity, which is open to the world and those interested in it, serves as a testbed for seeing how changes in robots synchronizing affects synchronization performance; if e.g. someone thinks they have an idea for making the robots more clever, they can easily download the Unity simulator, implement their approach, and see analytical and graphical (as well as audible) results.

Differently than to earlier approaches to synchronizing pulsed coupled oscillators, and harmonically at that \cite{nymoen_synch}, the synchronization system developed in this thesis in Unity does not limit oscillator frequencies to stay within a certain range, and oscillator frequencies can both be adjusted to be larger than the max initial frequency, as well as smaller than the minimum initial frequency. In e.g. Nymoen et al.'s firefly inspired oscillator system, frequencies $\omega$ are limited to staying within the range of $[0.5Hz, 8Hz]$.

% --- \section END. ---