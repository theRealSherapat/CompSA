\chapter{Introduction}
\label{chap:introduction}	


\section{Motivation}
Grunnene til å studere effektene av selv-bevissthet. \nl

% <<< \section END.



\section{Goal of the thesis} % a.k.a. research goals.


	% >>> META INFO START:

	\besk{"to make the reader better understand what the thesis is about"—Jim, og "en rød tråd?"—Sigmund}

	\besk{Spesifikke mål \tcol[gray]{(goals/aims)} med master-oppgaven/-prosjektet. Hva jeg vil vise/demonstrere til folk \tcol[gray]{(leserne f.eks.)} om self-awareness \tcol[gray]{(SA)} og hvordan \tcol[gray]{(detaljene av beskrevet i \textbf{Implementation})} — noe jeg håper å enten bekrefte eller avkrefte med eksperimenter og resultater i \textbf{Experiments and results}}

	% <<< META INFO STOP.


The specific aim of this thesis is to explore and investigate the effects of endowing robot systems with increased self-awareness abilities, and thus leads to the following research questions: \nl

\textbf{Research Question 1}:

Will performance in collective multi-robot systems increase as the level of \textit{self-awareness} increases? Specifically, will increased levels of self-awareness in the individual agents/musical robots lead to the collective of individuals being able to synchronize to each other faster than with lower levels of self-awareness? \nl

\textbf{Research Question 2}:

Will increased levels of self-awareness lead to more robustness and flexibility in terms of handling environmental noise and other uncertainties — specifically in the continued ability of musical robots to synchronize to each other efficiently despite these difficulties/challenges? \nl

% <<< \section END.



\section{Outline}
En fin \tcol[gray]{(Eagle's-eye)} oversikt over strukturen til hele dette dokumentet fra nå av, og utover.

% <<< \section END.



\section{Scope and delimitations}


	% >>> META INFO START:

	\besk{Hva oppgaven forsøker å oppnå eller besvare, og hva den ikke forsøker å oppnå eller besvare}

	% <<< META INFO STOP.



% <<< \section END.



\section{Contributions}


	% >>> META INFO START:

	\besk{De viktigste bidragene av MSc thesis-arbeidet summert \tcol[gray]{(for å få det til å stå frem/ut bedre enn å bare "gjemme" det i den siste delen av oppgaven)}. Fra Mats: HUSK Å SELG!}

	% <<< META INFO STOP.


This thesis work has led to some main novel contributions. The main contribution is that a novel synchronization-simulator has been created and developed in Unity, where various and customly made synchronization-methods can be implemented in a robot collective (either homogenously but also heterogenously in each robot); as well as that users of the simulator can both see and hear the robots's interactions and entrainment towards synchrony (including whether they manage to finally achieve synchrony, and in that case how long it takes them to get there). This novel framework opens, for everyone interested in it, for the possibility of experimenting with creative and novel synchronization-methods, in order to qualitatively and quantitatively assess their efficiency at achieving synchrony. A reason for this is due to the seamless scalability and dynamism, both in terms of robot-collective size, heterogenous or homogenous agents, as well as how a new update-function (for both phase-updating and frequency-updating) can so easily be added in and coded with a couple of lines.

% <<< \section END.