\chapter{Conclusions} % eller 'Discussion' som f.eks. Tønnes brukte.
\label{chap:conclusions}


	% --- META INFO START: ---

	\besk{ Where I shall follow-up on my \textit{research questions} by a discussion of to which degree—and in what ways—the thesis-/project-work has answered them }

	\gjor{
		Se på \tcol[gray]{(for kapittel-inspirasjon)}:
		\begin{itemize}
			\item Tønnes . MSc-thesis . Discussion-Ch.
			\item Jim . 'how to write a master thesis.pdf' . 'Conclusions'.
		\end{itemize}
	}

	% --- META INFO STOP. ---


\section{General discussion}



\section{Conclusion}



\section{Further work}

\textbf{Automatic and online (harmonic) fire sound generation}. As alluded to in \ref{human_audible_fire_signals}, one possible avenue to explore in order to find harmonious chords or tones—when played together—in a more automatic approach, live and online during simulation, could be found through musical machine learning models predicting fitting and accompanying chords and harmonies (given a tone) in real-time, like the ones tested thoroughly in B. Wallace's interactive music system PSCA \cite{wallace_PSCA}. This was however not attempted in this thesis, as this was beyond the scope of the project.

\textbf{Physical domain}. Going from simulation to the real physical world, using M. J. Krzyzaniak and RITMO's  actual musical robots, the Dr. Squiggles (fotnote) \cite{dr_squiggles}. Using an audio interface with the Dr. Squiggles robots in order to transmit or play audible ``fire'' signals would perhaps correspond to \textit{self awareness scope 2} as defined in this thesis, whereas \textit{self awareness scope 3} could possibly correspond in reality to communication technologies where messages are heard globally relatively instantaneously after it is transmitted, such as e.g. wireless local area networks (e.g. Wi-Fi) or—even faster—Ethernet internet cables.

\textbf{Hindrances throughout the thesis work}
The main reason why these aforementioned ideas were not possible to explore throughout this thesis work in the end, was due to an (to me at least) unseen and—for months—unobservable Unity peculiarity (perhaps only to a Unity novice like me) which in turn caused visible bugs in two mechanisms—namely the harmonic synchrony detection mechanism as well as my implementation of K. Nymoen et al.'s frequency adjustment method. So given that I only saw problems in the harmonic synchrony detection and frequency synchronization, I ended up spending considerable time throughout the thesis period trying to debug these two mechanisms, without success—or if any then with more bugs popping up. The way out of this extremely confusing and frustrating situation reared its head eventually as my attention was brought to the synchrony simulator's determinism. Little did I know that once I made my synchrony simulator deterministic, the problems I had been debugging for months were solved immediately. And after discussing with supervisors, and a much more experienced Unity user (Frank Veenstra), the root (and to me unseen) problem, and the reasons why it lead to those visible problems in my harmonic synchrony detection and frequency synchronization, suddenly made total sense. And so now this resulting knowledge about indeterminism in Unity, learnt the hard way, will most likely be shared through a ROBIN wiki page or the likes. Hopefully, this will prevent future UiO robotics master's students from stumbling onto the same problems I did, and from ending up in similar rabbit holes as I ended up in.