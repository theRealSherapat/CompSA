\chapter{Conclusions} % eller 'Discussion' som f.eks. Tønnes brukte.
\label{chap:conclusions}


	% --- META INFO START: ---

	\besk{ Where I shall follow-up on my \textit{research questions} by a discussion of to which degree—and in what ways—the thesis-/project-work has answered them }

	\gjor{
		Se på \tcol[gray]{(for kapittel-inspirasjon)}:
		\begin{itemize}
			\item Tønnes . MSc-thesis . Discussion-Ch.
			\item Jim . 'how to write a master thesis.pdf' . 'Conclusions'.
		\end{itemize}
	}

	% --- META INFO STOP. ---


\section{General discussion}

	As we can see from the results in the previous chapter, musical robot collectives achieve harmonic synchronization with various performance dependant on various robot collective or individual hyperparameter values.
	
	We see e.g. the recurring trend that the larger the robot collectives R we have (i.e. the larger $|R|$ we have), the more stable the robot collective seems to manage synchronizing to each other—compared to with lower collective sizes $|R|$.
	
	Perhaps not so surprisingly, performance both in terms of harmonic synchronization time as well as low error rates also tends to improve as we endow individual musical robots with larger self awareness scopes; the more of the closest neighbourhood each robot is self aware of, the higher the performance tends to be. However, there were also cases where not much improvement was seen in harmonic synchronization performance past certain points in self awareness scopes—even before robots having all their neighbours in their scope (i.e. global SA scope).


	\subsection{Conclusion}

	To summarize, time synchronization (harmonic at that) in musical robots, modelled as pulse coupled oscillators, has been achieved in the newly developed synchronization simulator, utilizing some already existing methods for synchronizing both oscillator phases and oscillator frequencies.
	
	Although individual hyperparameters plays a role in terms of harmonic synchronization performance to a certain extent, what is often more decisive for in a stable manner achieving harmonic synchrony in an oscillator collective is the collective size $|R|$.
	
	Furthermore, musical robots need not to be globally connected in order to achieve harmonic synchronization often times, and one can even relax the connectivity whilst still achieving as good synchronization performance as globally connected robots. 
	
	However, there still is an overarching trend of increasing harmonic synchronization performance the larger the self awareness scope of the individual musical robots are. Such increased degree of \textit{neighbour awareness} leading to higher performance in a multi agent system was also found by J. Cao et al. \cite{LINDA} as they managed to increase cooperation in a multi-agent reinforcement learning (MARL) scenario in the challenging videogame StarCraft II. This they achieved by increasing each agent's local awareness of its neighbour. To put this finding into a broader context, and a bit outside of what is studied here, this trend of increased cooperation and overall performance through \textit{increased neighbour self awareness} might be a good argument for why one should always be open to input—both in life, organizations, and businesses—as well as widen ones horizon and to be open minded. As Jordan Peterson's ninth rule in his book \textit{12 Rules for Life} reads: ``Assume that the person you are listening to might know something you don't."


	\subsection{Ethical reflections}
	
	One might get frightened by hearing robots or machines have become self aware. This can remind listeners of concepts like superintelligence or general artificial intelligence, which can have scary and unknown consequences if ever achieved.
	
	My personal view is that the work presented in this thesis probably holds potential for computing systems and robots to get more well defined and consise self awareness specifications, and hence might become more self aware as a consequence of such clarity. Furthermore, frameworks for computing systems inspired by psychology in terms of self awareness might potentially get us closer to achieving something resembling general artificial intelligence or a superintelligence. However, I do not fear of this happening in the near future or even the far future; I would like to echo what Andrew Ng, being one of the leading figures within the AI community, has been quoted to say regarding the possibility of an AI superintelligence: ``Worrying about AI evil superintelligence today is like worrying about overpopulation on the planet Mars. We haven’t even landed on the planet yet!'' I, like others, even think there is a safety argument to be had about self awareness having a larger focus in computational systems like robots, being that explainability (being a hot topic at the moment) can be further increased by considering computational self awareness in a computing system. However, I am very in favor of careful considerations and regulatory oversight from all parts of society in order for humans to ensure control and desired outcomes when it comes to both AI, and more specifically to topics like the one explored in this thesis being computational self awareness.


\section{Further work}

	\paragraph{Physical domain:}
	Going from simulation to the real physical world, using M. J. Krzyzaniak and RITMO's  actual musical robots, the Dr. Squiggles (cf. \ref{dr_squiggles}). This would involve implementing or translating the software written for the simulated Dr. Squiggles in the Unity simulator environment into the physical hardware and interfaces available in the real Dr. Squiggle robots. Using an audio interface with the Dr. Squiggles robots in order to transmit or play audible ``fire'' signals would perhaps correspond to \textit{self awareness scope 2} as defined in this thesis, whereas \textit{self awareness scope 3} could possibly correspond in reality to communication technologies where messages are heard globally relatively instantaneously after it is transmitted, such as e.g. wireless local area networks (e.g. Wi-Fi) or—even faster—Ethernet internet cables.

	\paragraph{Introduction of noise and epistemolgical challenges:}
	In this music system, pulses (or firings) were designed in such a way that the other musical nodes would easily hear the distinct signals. We could have easily imagined a scenario where this would not be given; hence calling for more sophisticated methods for detecting and separating between actually separate (spatially) audio-sources. Epistemological explorations, related to e.g. Computational Auditory Scene Analysis (as mentioned in \cite{casa}) would then be of interest. Some simple ways to introduce noise in the developed synchronization simulator would be to omit or skip, with a probability, notifying a robot by sending a ``fire'' or adjustment signal to it for self adjustment. In this way, it would be like robots randomly not managing to listen for all the ``fire'' signals being transmitted into the environment. One could then do a study on how well different musical robot collectives, in terms of e.g. self awareness scopes, would handle this additional challenge.
	
	% \paragraph{Learning of internal models capturing self awareness knowledge:}
	

	\paragraph{Automatic and online (harmonic) fire sound generation:}
	As alluded to in \ref{human_audible_fire_signals}, one possible avenue to explore in order to find harmonious chords or tones—when played together—in a more automatic approach, live and online during simulation, could be found through musical machine learning models predicting fitting and accompanying chords and harmonies (given a tone) in real-time, like the ones tested thoroughly in B. Wallace's interactive music system PSCA \cite{wallace_PSCA}. This was however not attempted in this thesis, as this was beyond the scope of the project.

	\paragraph{Hindrances throughout the thesis work:}
	The main reason why these aforementioned ideas were not possible to explore throughout this thesis work in the end, was due to an (to me at least) unseen and—for months—unobservable Unity peculiarity (perhaps only to a Unity novice like me) which in turn caused visible bugs in two mechanisms—namely the harmonic synchrony detection mechanism as well as my implementation of K. Nymoen et al.'s frequency adjustment method. So given that I only saw problems in the harmonic synchrony detection and frequency synchronization, I ended up spending considerable time throughout the thesis period trying to debug these two mechanisms, without success—or if any then with more bugs popping up. The way out of this extremely confusing and frustrating situation reared its head eventually as my attention was brought to the synchrony simulator's determinism. Little did I know that once I made my synchrony simulator deterministic, the problems I had been debugging for months were solved immediately. And after discussing with supervisors, and a much more experienced Unity user (Frank Veenstra), the root (and to me unseen) problem, and the reasons why it lead to those visible problems in my harmonic synchrony detection and frequency synchronization, suddenly made total sense. And so now this resulting knowledge about indeterminism in Unity, learnt the hard way, will most likely be shared through a ROBIN wiki page or the likes. Hopefully, this will prevent future UiO robotics master's students from stumbling onto the same problems I did, and from ending up in similar rabbit holes as I ended up in.