\chapter*{Abstract}
\label{abstract}

Engineering a computing system for a certain environment often requires some knowledge of said environment; both on the end of the system-creator, as well as for the computing system in turn. This knowledge will then have to be used in the environment the computing system is situated in, in all the situations and possible states the system can be in during its lifetime. However, foreseeing and predicting, at design-time, all possible future states and scenarios a computing system will be in during its lifetime is often hard, and sometimes impossible (in the case of coincidental faults e.g.). Increasingly extreme, complex, dynamic, and ever-changing environments, simply enlargens this challenge. This calls for online and continuous learning at run time, where computing systems are themselves able to observe, learn, adapt, and act on their own, independently from their creator.

Especially within collectives of multiple robots \cite{cocoro, swarm_bot}, communication between individuals—even though they might be autonomous individual robots—is still a challenge one has to deal with when designing them. Coordination, for humans and robots alike, often tend to demand a lot of communication. To add on top of this, when networks or collectives of sub-systems like wireless sensors are communicating with each other, they often have their own internal clock which often do not align. Hence, we see that both the magnitude of communication needed, as well as the inherent challenges with communication, call for effective approaches to communication with coordination as the goal.

As K. Konishi \& H. Kokame \cite{konishi_kokame} points out, one technical problem in e.g. wireless sensor networks is that each sensor node rely on accurate internal and synchronized clocks, especially from the perspective of sensor fusion and coordinating communication among nodes. Thus, time synchronization becomes recognized as one of the crucial problems in distributed and wireless systems \cite{tungvinte_sync_protocols}. Various attempts at synchronizing messages and communication, in order to e.g. align messages exchanged at unsynchronized intervals, have been made \cite{tungvinte_sync_protocols}; however, such attempts and protocols often require the computation of message exchange and processing, which wastes the limited computation capability of nodes and causes communication delays. To this, Konishi \& Kokame \cite{konishi_kokame} also point out how for synchronized pulse coupled oscillators (PCOs), such computation is not required. If internal clocks in communicating nodes are instead altered and synchronized to other nodes's clocks—instead of letting all clocks stay unsynchronized and constantly trying to compensate for this— it is rather apparent that getting rid of this need for compensation will considerably reduce computation needs while synchronized communication is still achieved. Hence, by achieving time synchronization, both the computational load, in addition to the need for communication, is considerably reduced.

Thus, by achieving time synchronization \cite{tungvinte_sync_protocols} through synchronizing pulse coupled oscillators, both the computational load, in addition to the need for communication, is considerably reduced; hence freeing up and saving valuable resources like e.g. processing power and energy (battery) in networks of individuals and possibly autonomous nodes, agents, or robots.

Despite its elusive, and at times hard-to-tangibly-define, nature; the psychological concept of Self-Awareness has relatively recently served as a rich source of new inspiration and conceptual tools and frameworks, previously unknown or undiscovered by the computer engineer.

With self awareness, online and continuous learning is achieved to a higher degree in contrast with other approaches (like ODA and MAPE-K), due to the limitations and downsides of these older approaches, as well as the advantages and upsides to considering computational self-awareness in computing systems. If one wants to achieve continuous adaptation of a system or of system-components (e.g. in a collective) — some sort of intelligence might be necessary to endow it with. Endowing computing systems with Self-Awareness can be beneficial in several respects, including but not limited to a greater capacity to adapt, to build potential for future adaptation in unknown environments, and to explain their behaviour to humans and other systems \cite{sacs17_ch3}.

Taking inspiration from the fascinating natural phenomena of self synchronizing fireflies, a synchronization simulator imitating and modelling this process for—instead of fireflies—a collective of musical robots, has been designed and tested. It has been found that e.g. the sensitivity of the robots when it comes to adjusting themselves on the way to synchronizing to each other does not have as much a say on the stability nor performance of the synchronization task as e.g. the robot collective size does. Furthermore, it is also found that the synchronization task can still be achieved despite more relaxed connectivity in the network of robots.

We found for a synchronization scenario that when robots listen to 40\% of their closest neighbours, they achieve just as good performance as globally connected robots do.