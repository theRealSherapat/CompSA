\section{Introduction}
	% Here's where my journey begins. I have Lucas Paruch and Viktoria Stray's wish of "good and best of luck!" (except that I don't believe in luck)
	
	% CONSIDER: HOW MUCH OF THIS BELONGS HERE, AND WHAT BELONGS IN THE Abstract?
	% Subsection 1.1
	\subsection{Motivation}
	
	\gjor{Se om det er noe fra essayet og kommentarene i .tex-fila derfra som jeg vil gjenbruke her}
	
	\gjor{Tenk igjennom og besvar så godt som mulig etterhvert (kompilert fra Samuelsens MSc-thesis. Kan cross-checke med Tønnes og hun andres også):
		\subitemize{\tbf{Why is the thesis topic, and its outflowing proposed solutions/improvements, of relevance in the world today?}
					\begin{itemize}
						\item History of field, how things have been done before — and why the situation/needs/requirements might have changed, or why these traditional/typical solutions may be ripe for improvements or better solutions? Why are these concerns/problems/factors of importance?
							\subitemize{Demonstrate, illustrate, and explain these changes / this new situation so that the reader understands why your topic's contributions are necessary or needed.}
						\item What are the relevant real-world problems in need of solutions/improvement, where the thesis topic can provide such solutions/improvements?
						\item Differentiate between what the "Background-/Related-works-proposed method" conributes with, and the "new proposed method" that you yourself want to try out (e.g. differentiate between ODA-loops and MAPE-K-loops, and endowing computational systems with \tit{computational self-awareness} (and \tit{self-expression}).
							\subitemize{Explain why the "new proposed method" is needed/granted, maybe in relation to a lack or challenge with the original "Background-/Related-works-proposed method". Perhaps also mention the absence or "freshness" of this "new proposed method" in the history or field of the "Background-/Related-works-proposed method".}
					\end{itemize}
				}
	}
	
	\separate
	
	\inkl{Engineering a computing system for a certain environment often requires some knowledge of said environement — both on the end of the creator of the computing system, as well as for the computing system in turn. This is at least the case in autonomous computing, where computing systems are supposed to be able to observe, learn, adapt, and act on their own — independently from their creator.}
	\nl
	
	However, predicting all possible future states of complex, dynamic, and ever-changing environments is hard, and at times impossible. \inkl{This calls for online and continuous learning, don't you think? How to best tackle this problem? Glad you asked. — With Self-Awareness of course. Because ...}
	\nl
	
	
	
	
	
	
	
	% Subsection 1.2
	\subsection{Goal of the thesis}
	% Research Goals / Goal of the thesis:
	\gjor{Kople tekst oppmot Research-Spørsmålene mine her}
	
	
	
	
	
	
	
	
	% Subsection 1.3
	\subsection{Outline}
	
	\gjor{Skriv opp strukturen/oversikten (Eagle's-eye) av thesis-dokumentet her}
