\section{Introduction}
	% Here's where my journey begins. I have Lucas Paruch and Viktoria Stray's wish of "good and best of luck!" (except that I don't believe in luck)
	
	% CONSIDER: HOW MUCH OF THIS BELONGS HERE, AND WHAT BELONGS IN THE Abstract?
	% Subsection 1.1
	\subsection{Motivation}
	
	\gjor{Tenk igjennom og besvar så godt som mulig etterhvert (kompilert fra Samuelsens MSc-thesis, og etterhvert Essay-kommentarer. Kan cross-checke med Tønnes og hun andres også):
		\subitemize{
			\textbf{Why is the thesis topic, and its outflowing proposed solutions/improvements, of relevance in the world today?}
				\begin{itemize}
					\item History of field, how things have been done before — and why the situation/needs/requirements might have changed, or why these traditional/typical solutions may be ripe for improvements or better solutions? Why are these concerns/problems/factors of importance?
						\subitemize {
							Demonstrate, illustrate, and explain these changes / this new situation so that the reader understands why your topic's contributions are necessary or needed.
						}
					\item What are the relevant real-world problems in need of solutions/improvement, where the thesis topic can provide such solutions/improvements?
					\item Differentiate between what the "Background-/Related-works-proposed method" conributes with, and the "new proposed method" that you yourself want to try out (e.g. differentiate between ODA-loops and MAPE-K-loops, and endowing computational systems with \textit{computational self-awareness} (and \textit{self-expression}).
						\subitemize {
							Explain why the "new proposed method" is needed/granted, maybe in relation to a lack or challenge with the original "Background-/Related-works-proposed method". Perhaps also mention the absence or "freshness" of this "new proposed method" in the history or field of the "Background-/Related-works-proposed method".
						}
					\lab{From Essay-comments} {
						\item What constitutes the essential ideas behind the solutions to the problem?
					
						\item Have I written a short description of the problem / challenge for my thesis?
						
						\item Have I introduced (understandably and intuitively) Self-Awareness as an exciting and relevant field/source-of-inspiration-and-concepts?
						
						\item Have I presented motivations and arguable advantages of endowing a computational system with Self-Awareness (Se reMarkable'n)?
						
						\item Have I discussed or argued for the motivations for endowing, especially and in particular, music systems with \textit{Computational Self-Awareness} — and then connected this with efforts like Nymoen et al.'s Firefliy-synchronization and/or Chandra et al.'s Solojam?
					}
				\end{itemize}
			}
	}
	
	\sep
	
	\lab{Fra Essay-Introduction} {
		\inkl {
			Designing and predicting all possible scenarios of a computing system at design-time is often hard, and sometimes impossible (in the case of unpredictable faults e.g.). If one wants to achieve coordination and continuous adaptation of a system or of system-components (for example in a collective) -- some sort of intelligence might be necessary to endow it with. Endowing computing systems with Self-Awareness can be beneficial in several respects, including a greater capacity to adapt, to build potential for future adaptation in unknown environments, and to explain their behaviour to humans and other systems [SACS 17 Ch. 3]. Self-awareness concepts from psychology are inspiring new approaches for engineering computing systems which operate in complex dynamic environments [SACS 16 Ch. 2]. As we can see in various Music Technology Systems, this endowing can also give rise to interesting cooperative and coordinating behaviour.
		}
		
		\inkl {
			The problem of this thesis will mainly consist of studying the effects differing Self-Awareness levels, varying collective-sizes, levels of task difficulty (like more complex behaviours, and limited communication) -- have on usefulness, system dynamics, overall performance, and scalability (primarily within the domain of Musical Multi-Robot/-Agent collectives).
		}
	}
	
	\lab{Fra Essay-Introduction-kommentarer} {
		\inkl {
			In this MSc. thesis, we will explore an exciting and relatively new translation of the concepts and notions regarding \textit{self-awareness} -- as they pertain to humans and animals especially -- from the domain of Psychology, into the domain of Computation and Engineering.
		}
		
		\inkl {
			We will in this project attempt to implement and explore whether, and indeed in what way (how), a computational system (like computational agents/agent-collectives in a computer-simulation, and/or even on physical musical robots) can exibit and display Self-Awareness (and corresponding Self-Expressive) capabilities, comparable to how they are perceived in humans and animals.
		}
		
		\inkl {
			We will in this thesis project investigate methods and models for computational self-awareness in multi-robot systems, with application to the musical robotics domain.
		}
	}
	
	\inkl{Engineering a computing system for a certain environment often requires some knowledge of said environement — both on the end of the creator of the computing system, as well as for the computing system in turn. This is at least the case in autonomous computing, where computing systems are supposed to be able to observe, learn, adapt, and act on their own — independently from their creator.}
	\nl
	
	However, predicting all possible future states of complex, dynamic, and ever-changing environments is hard, and at times impossible. \inkl{This calls for online and continuous learning, don't you think? How to best tackle this problem? Glad you asked. — With Self-Awareness of course. Because ...}
	\nl
	
	
	
	
	
	
	
	% Subsection 1.2
	\subsection{Goal of the thesis}
	% Research Goals / Goal of the thesis:
	\gjor{Kople tekst oppmot Research-Spørsmålene mine her}
	
	
	
	
	
	
	
	
	% Subsection 1.3
	\subsection{Outline}
	
	\gjor{Skriv opp strukturen/oversikten (Eagle's-eye) av thesis-dokumentet her}
