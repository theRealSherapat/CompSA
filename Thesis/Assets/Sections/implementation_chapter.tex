\chapter{Implementation}
	% IFØLGE JIM ER DET HER FISKEN HENTES TILBAKE PÅ LAND (BLANT ANDRE STEDER).
	% SE PÅ TØNNES SIN MASTEROPPGAVE FOR INSPIRASJON.
	\section{\tcol[gray]{Betraktninger}}
		\besk{"Det man har utviklet" — Kyrre}

		\gjor{Skriv opp Worklog-materiale dandert i henhold til gode master-theses}
	
	
	
	\section{Benchmark}
		\besk{Her jeg beskriver Nymoens algoritmer og formler (originalt), men \powof{sånn jeg har implementert det i Unity}}
	
		\gjor{Vurder om 'Benchmark' bør deles opp i sin egen sub-seksjon i det hele tatt — eller f.eks. slås sammen med 'Proposed Algorithm'}
	
		\besk{(Hentet fra Samuelsens master?) Presentering av metoden brukt til å evaluere ytelsen av den foreslåtte/proposed'e algoritmen. Først er kanskje en referanse-algoritme brukt for sammenlikning beskrevet. Deretter er (f.eks. objektiv-) funksjoner brukt i testene forklart. Endelig (til slutt) er kanskje miljøene/environments'a og parameterne brukt presentert}
		
		
		
	\section{Proposed Algorithm}
		\besk{Evt. her jeg skriver om Self-Awareness-komponenten(e) jeg legger til ang. Belief-awareness og/eller Expectation-awareness (jf. det jeg og Kyrre snakka om Mid-November på 'reMarkable -> Møter -> ROBIN -> Kyrre')}
	
		\gjor{Vurder om 'Proposed Algorithm' bør deles opp i sin egen sub-seksjon i det hele tatt — eller f.eks. slås sammen med 'Benchmark'}
	
		\besk{(Hentet fra Samuelsens master?) Presentering av metoden brukt til å evaluere ytelsen av den foreslåtte/proposed'e algoritmen. Først er kanskje en referanse-algoritme brukt for sammenlikning beskrevet. Deretter er (f.eks. objektiv-) funksjoner brukt i testene forklart. Endelig (til slutt) er kanskje miljøene/environments'a og parameterne brukt presentert}
		