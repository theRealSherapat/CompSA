\chapter{Introduction}
	% Here's where my journey begins. I have Lucas Paruch and Viktoria Stray's wish of "good and best of luck!" (except that I don't believe in luck)
	
	
	
	
	
	
	
	
	% CONSIDER HOW MUCH OF THIS BELONGS HERE, AND WHAT BELONGS IN THE Abstract.
	% Subsection 1.1
	\section{Motivation}
	
	Grunnene til å studere effektene av selv-bevissthet.
	\nl
	
	De viktigste bidragene av MSc thesis-arbeidet summert \tcol[gray]{(for å få det til å stå frem/ut bedre enn å bare "gjemme" det i den siste delen av oppgaven)}.
	
	
	
	
	
	
	
	
	
	% Subsection 1.2
	\section{Goal of the thesis}
	% Research Goals / Goal of the thesis:
	\besk{"to make the reader better understand what the thesis is about"—Jim, og "en rød tråd?"—Sigmund}
	
	Spesifikke mål \tcol[gray]{(goals/aims)} med master-oppgaven/-prosjektet. Hva jeg vil vise/demonstrere til folk \tcol[gray]{(leserne f.eks.)} om self-awareness \tcol[gray]{(SA)} og hvordan.
	\nl
	
	\textbf{Research Question 1}:
	
	Will performance in collective multi-robot systems increase as the level of Self-Awareness increases? Specifically, will increased levels of Self-Awareness in the individual agents/musical robots lead to the collective of individuals being able to synchronize to each other faster than with lower levels of Self-Awareness? \nl
	
	\textbf{Research Question 2}:
	
	Will increased levels of Self-Awareness lead to more robustness and flexibility in terms of handling environmental noise and other uncertainties — specifically in the continued ability of musical robots to synchronize to each other efficiently despite these difficulties/challenges? \nl
	
	
	
	
	
	
	
	% Subsection 1.3
	\section{Outline}
	En fin \tcol[gray]{(Eagle's-eye)} oversikt over strukturen til hele dette dokumentet fra nå av, og utover.
