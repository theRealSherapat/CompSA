\chapter{Implementation}
\gjor{Skriv opp Worklog-materiale (under ``muy bien so far''-bokmerket) dandert i henhold til gode master-theses}


% Introducing the Chapter
This chapter gives an overview of the developed musical multi-robot system. The main goal of the implemented system is to allow for a multi-robot (musical) collective to interact with each other in order to achieve emergent and co-ordinating/co-operative behaviour—synchronization specifically in our case—with varying degrees of difficulty and certainty in the environment and communication. More specifically, the goal with the design is to enable the robot collective to achieve so-called \textit{harmonic synchronization} within a relatively short time. What is meant by \textit{harmonic synchronization} will be expounded in Subsection \ref{subsec:harmonic_synchrony}. These goals firstly require of the agents/nodes the modelling of oscillators with their properties, like phase and frequency, as explained further in Subsection \ref{subsec:node}. To allow for interaction and communication between the agents, mechanisms so that the agents can transmit "fire"-signals, as well as listen for other agents's "fire"-signals, is necessary as well, and is presented in Subsection \ref{subsec:fire_signal}.



% Om baselinen (hva jeg baserer min implementasjon på, forskjellig fra mitt eget bidrag).
% Introducing the System:
\section{The baseline: Achieving Decentralized Harmonic Synchronization in (Oscillators $\vee$ Musical Robot Collectives)}
\label{sec:baseline}
	Envision that we have a multi-agent collective scenario consisting of musical robots modelled as oscillators, solely communicating through brief ``fire''-like audio-signals|greatly inspired by K. Nymoen et al.'s approach for achieving \textit{decentralized harmonic synchronization in mobile music systems} \cite{nymoen_synch}. These agents are initially not synchronized in their firing of audio-signals, but as time goes, they are entraining to synchronize to each other by adjusting their phases and frequencies when or after hearing each other's audio-signals. An illustration of this is given in Figure \ref{fig:initial}.

	% First Intro-illustration figure to easily get a quick idea of what the system/design does/consists of:
	\begin{figure}[h]
		\centering
			\begin{subfigure}[t]{.5\textwidth}
				\centering\captionsetup{width=.9\linewidth}%
				\includegraphics[width=0.9\linewidth]{Assets/Figures/IntroUnsynch.jpg}
				\caption{The agents firing asynchronously at first. Here, only the two Dr. Squiggles with red tentacles are firing simultaneously, but the rest are not.}
				\label{fig:initial:unsynch}
			\end{subfigure}%
			\begin{subfigure}[t]{.5\textwidth}
				\centering\captionsetup{width=.9\linewidth}%
				\includegraphics[width=0.9\linewidth]{Assets/Figures/IntroSynch.jpg}
				\caption{Seconds later, after having listened to each other's fire-event signals and adapted themselves accordingly, the agents are here firing synchronously.}
				\label{fig:initial:synch}
			\end{subfigure}
		\caption{Decentralized Synchronization of phases achieved in a musical robot collective, consisting of M. J. Krzyzaniak and RITMO's Dr. Squiggles.}
		\label{fig:initial}
	\end{figure}

	These aforementioned audio-signals to be expounded further in Subsection \ref{subsec:fire_signal}, also referred to as ``fire''-signals, are transmitted whenever an agent's oscillator \textit{peaks} (i.e. after its cycle or period is finished, having phase $\phi(t)=1$)|or actually every second \textit{peak}, due to the target system goal of \textit{harmonic synchrony}. All agents have the ability to listen for such transmitted ``fire''-signals from their neighbours, which they then will use as a trigger to adjust themselves according to some well-designed update-functions to be elaborated in Subsection \ref{subsec:update_functions}.

	% INCLUDE? See 'Relevant MSc-thesis Concerns'.
		% % Phase-/time-plot Figure with two Subplots. Subplot 1: phase-/time-plot when oscillators are unsynchronized. Subplot 2: phase-/time-plot when oscillators are synchronized
		% \begin{figure}[h]
			% \centering
				% \begin{subfigure}[t]{.5\textwidth}
					% \centering\captionsetup{width=.9\linewidth}%
					% \includegraphics[width=0.9\linewidth]{Assets/Figures/phases_unsynched.png}
					% \caption{Oscillators are initially not (phase-) synchronized.}
					% \label{fig:phases_unsynched}
				% \end{subfigure}%
				% \begin{subfigure}[t]{.5\textwidth}
					% \centering\captionsetup{width=.9\linewidth}%
					% \includegraphics[width=0.9\linewidth]{Assets/Figures/phases_synched.png}
					% \caption{Oscillators are now, seconds later, (phase-) synchronized.}
					% \label{fig:phases_synched}
				% \end{subfigure}
			% \caption{Phase-plots for the agents}
			% \label{fig:initial}
		% \end{figure}



	% Sub-Introducing the System:
	
	% Om den enkelte noden/agenten med alle egenskaper den har osv.
	\subsection{The Node: the musical robot individual}
	\label{subsec:node}
		\besk{Om den enkelte noden/agenten med alle egenskaper den har osv. (som f.eks. en oscillator-komponent (jf. I.A., III.Intro., og 'Implementation' i Nymoens Firefly-paper))}


	% Om kommunikasjonen til agentene: audio-/``fire''-signalet
	\subsection{Robot communication: the ``fire''-signal}
	\label{subsec:fire_signal}
		\begin{itemize}
			\item Short and impulsive audio sound/signal, representing a node's phase-climax and ``fire''-/``flash''-event.
			\item The only means of communication — allowing for stigmergic co-ordination.
			\item Facilitating PCO (pulse-coupled oscillators), different from phase-coupled oscillators with their differences.
		\end{itemize}


	% Om target-staten til systemet: harmonisk synkroni
	\subsection{System target state: Harmonic Synchrony}
	\label{subsec:harmonic_synchrony}
		\begin{itemize}
			\item The goal and target state of the system
			\item All agents/node having integer-relation frequencies, so that given a fundamental frequency $\omega_{i,0}$ for the agent $i$ with the lowest frequency in the collective — all other agents have "legal" harmonic frequencies; meaning, all nodes have frequencies in the \tcol{mengde} $\omega_{i,0} \cdot 2^{\mathbb{N}_0}$.
		\end{itemize}
	
	
	% Om Update-functions for frekvens- og fase-oppdateringene
	\subsection{Update/Adjustment functions: Phase- \& Frequency-Adjustment}
	\label{subsec:update_functions}
	
	\lab{Sigmund Stue-(post IN3190-eksamen)Recall}{
		\textbf{S}: Oppdateres først faser helt ferdig først — og så begynner frekvenser å oppdatere seg? Eller hvordan oppdateres egentlig frekvensen (hva er formelen for det, sammenliknet med den nokså enkle formelen til fase-oppdateringen)? \nl
		
		\textbf{D}: Nei, fase og frekvens oppdateres samtidig, men en nodes fase oppdateres umiddelbart når noden hører et ``fire''/``flash''-signal — til forskjell fra nodens frekvens. For frekvens-justering, samler noden opp (akkumulerer) (error-) scores gjennom hele sin periode/syklus. Hver (error-) score sier noe om hvor mye \textit{ute av synch} noden er, og da altså hvor mye frekvensen dens bør oppdateres for å komme inn i synch (små error-scores fører da til lite behov for justering, og frekvensen oppdateres kun litt). Noen tilleggs-operasjoner på frekvens-oppdaterings-funksjonen forteller også noe om hvilket fortegn denne frekvens-oppdateringen bør ha (da frekvens-justeringen, i likhet med fase-justeringen, også er bi-direksjonell). Gjennomsnittet av disse error-scoresene multiplisert med fortegns-og amplitude-faktorene $\rho(n)$, vektet av en \textit{frekvens-koplings konstant} $\beta$, blir da brukt i kalkuleringen for hva den nye og oppdaterte frekvensen $\omega_i(t^+)$ blir, for agent $i$, der $t^+$ er tidssteget umiddelbart etter frekvensjusteringen er gjennomført
	}
	
		\subsubsection{Phase Adjustment}
		\label{subsubsec:phase_adj}
			If we first assume constant and equal oscillator-frequencies in our agents, we can take a look at how the agents adjust their phase in order to synchronize to each other. This is in contrast to the case in \ref{subsubsec:freq_adj} where heterogenous, often randomly initialized, oscillator-frequencies in the musical agents are implemented and utilized.
			
			Two approaches were attempted in Unity, as presented in Nymoen et al.'s ``\textit{Firefly-paper}'' \cite{nymoen_synch}:
			
			\myparagraph{1) Mirollo-Strogatz's ``standard'' Phase Adjustment}
			
			\begin{figure}[h]
				\centering
				\includegraphics[width=0.9\linewidth]{Assets/Figures/MirolloStrogatzPhaseAdjustment.pdf}
				\caption{``Standard'' Phase-Adjustment with Mirrollo-Strogatz's approach}
				\label{fig:strog_phase}
			\end{figure}
			
			Each musical node updates its phase $\phi$ accoring to the \textbf{phase update function \eqref{strog_phase}} when hearing a ``fire''-event from one of the other musical nodes:
			
			\begin{equation}\label{strog_phase}
			P(\phi) = (1 + \alpha)\phi	,
			\end{equation}
			
			where ``\textit{$\alpha$ is the pulse coupling constant, denoting the strength between nodes}'' \cite{nymoen_synch}. So, if $\alpha = 0.1$ e.g., then a musical node's new and updated phase, immediately after hearing a ``fire''-signal from another node, will be equal to $P(\phi) = (1 + 0.1)\phi = 1.1\phi$. 110\% of its old phase $\phi$, that is. Hence, and in this way, the node would be ``pushed'' to fire sooner than it would otherwise. An illustration of this is given in Figure \ref{fig:strog_phase}.
			
			
			\myparagraph{2) Nymoen et al.'s Bi-Directional Phase Shifts}
			
			\begin{figure}[h]
				\centering
				\includegraphics[width=0.9\linewidth]{Assets/Figures/NymoenPhaseAdjustment.pdf}
				\caption{Bi-directional Phase-Adjustment with K. Nymoen et al.'s approach}
				\label{fig:nymoen_phase}
			\end{figure}
			
			This Phase-adjustment, as in Figure \ref{fig:nymoen_phase}, works very similarly to the Phase-adjustment performed in the \textbf{\textit{Mirollo-Strogatz}} approach; The only difference being that now, nodes update their phase with the slightly more complex \textbf{phase update function \eqref{nymoen_phase}} when hearing a ``fire''-event from one of the other musical nodes — allowing for both larger, but also smaller, updated phases:
			
			\begin{equation}\label{nymoen_phase}
			P(\phi) = \phi - \alpha \cdot sin(2\pi\phi) \cdot | sin(2\pi\phi) |
			\end{equation}
			
			The fact that new and updated phases can both be larger, but also smaller, compared to the old phases, is exactly what's meant by the Phase-adjustment being \textbf{\textit{Bi-Directional}}, or as the authors call it in the paper as using ``\textit{both excitatory and inhibitory phase couplings between oscillators}'' \cite{nymoen_synch}.
			
			The effects then of adjusting phases, upon hearing ``fire''-events, according to this newest update-function \eqref{nymoen_phase} are that the nodes's phases now get decreased if $\phi$ is lower than 0.5, increased if $\phi$ is higher than 0.5, and neither—at least almost—if the phases are close to $0.5 = \frac{1}{2}$. This is due to the negative and positive sign of the sinewave-component in Equation \eqref{nymoen_phase}, as well as the last attenuating factor in it of $| sin(2\pi\phi) | \approx | sin(2\pi \frac{1}{2}) | = | sin(\pi) | = | 0 | = 0.$
		
		
		
		\subsubsection{Frequency Adjustment}
		\label{subsubsec:freq_adj}
			\gjor{Presenter hva Frequency-adjustment er (hvis det ikke ble tydelig nok i slutten av Seksjon-\ref{sec:baseline}), bygg opp fremgangsmåten på hvordan man oppnår Frequency-adjustment (som i reMarkable-notatet 'Logs' $\rightarrow$ 'Simulations' $\rightarrow$ 'Frequency adjustment', Worklog'en, og Mattermost-chatten med Kyrre evt.) | med passende figurer og illustrasjoner}





% Om mitt bidrag (skill tydelig på mine bidrag og det jeg ikke har gjort noe med)
\section{My new proposed algorithm: an additional Self-Awareness component}
