\chapter{Introduction}
\label{chap:introduction}
	% Here's where my journey begins. I have Lucas Paruch and Viktoria Stray's wish of "good and best of luck!" (except that I don't believe in luck)
	
	
	
	
	
	
	
	
	% CONSIDER HOW MUCH OF THIS BELONGS HERE, AND WHAT BELONGS IN THE Abstract.
	% Subsection 1.1
	\section{Motivation}
	
	Grunnene til å studere effektene av selv-bevissthet.
	\nl
	
	De viktigste bidragene av MSc thesis-arbeidet summert \tcol[gray]{(for å få det til å stå frem/ut bedre enn å bare "gjemme" det i den siste delen av oppgaven)}.
	
	
	
	
	
	
	
	
	
	% Subsection 1.2
	\section{Goal of the thesis}
	% Research Goals / Goal of the thesis:
	\besk{"to make the reader better understand what the thesis is about"—Jim, og "en rød tråd?"—Sigmund}
	
	Spesifikke mål \tcol[gray]{(goals/aims)} med master-oppgaven/-prosjektet. Hva jeg vil vise/demonstrere til folk \tcol[gray]{(leserne f.eks.)} om self-awareness \tcol[gray]{(SA)} og hvordan.
	\nl
	
	The specific aim of this thesis is to explore and investigate the effects of endowing robot systems with increased self-awareness abilities, and thus leads to the following research questions:
	
	\textbf{Research Question 1}:
	
	Will performance in collective multi-robot systems increase as the level of Self-Awareness increases? Specifically, will increased levels of Self-Awareness in the individual agents/musical robots lead to the collective of individuals being able to synchronize to each other faster than with lower levels of Self-Awareness? \nl
	
	\textbf{Research Question 2}:
	
	Will increased levels of Self-Awareness lead to more robustness and flexibility in terms of handling environmental noise and other uncertainties — specifically in the continued ability of musical robots to synchronize to each other efficiently despite these difficulties/challenges? \nl
	
	
	
	
	
	
	
	% Subsection 1.3
	\section{Outline}
	En fin \tcol[gray]{(Eagle's-eye)} oversikt over strukturen til hele dette dokumentet fra nå av, og utover.


	\section{Scope and delimitations}

	\section{Contributions}
	This thesis work has led to some main novel contributions. The main contribution is that a novel synchronization-simulator has been created and developed in Unity, where various and customly made synchronization-methods can be implemented in a robot collective (either homogenously but also heterogenously in each robot); as well as that users of the simulator can both see and hear the robots's interactions and entrainment towards synchrony (including whether they manage to finally achieve synchrony, and in that case how long it takes them to get there). This novel framework opens, for everyone interested in it, for the possibility of experimenting with creative and novel synchronization-methods, in order to qualitatively and quantitatively assess their efficiency at achieving synchrony.