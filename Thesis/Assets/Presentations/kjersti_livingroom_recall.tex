\lab{Tante-Kjersti-inspirert {\normalsize (ish Stue-Recall)}}{
	The diverse and complex phenomena of nature have for long served as exciting inspirations to human engineering and research (cite ant colonies, boids \& swarms, beeclust e.g.). One such phenomena studied and attempted modelled is the synchronous firing of fireflies in the rainforests.
	\nl

	\gjor{Insert illustration/picture of synchronizing/synchronized fireflies firing in a dark forest here}

	This has inspired scientists like Mirollo \& Strogatz \cite{mirollo_strogatz_PCO_synch}, and in later time Kristian Nymoen, Kyrre Glette et al. \cite{nymoen_synch}, to attempt to model and "etterlikne" this natural phenomenon in human-engineered systems. This work ties into the work on synchronizing oscillators \opphoy{\cite{,,}} which has been subject to study for some time now. What separates Mirollo \& Strogatz and K. Nymoen's approach from these previous ones, is that here the oscillators are \textit{pulse-coupled}, as opposed to the more normal and constraining \textit{phase-coupled} (\opphoy{explain}).
	Each modelled "firefly", or firing node, is here implemented and considered as an oscillator, characterized by its phase and frequency. \inkl{Kinda, the job is to align sinusoidal waves, either by shifting an agent's phase "up", or "down".}
	
	\inkl{No training of any neural networks or any model-data was needed to achieve synchrony in this case — and so far no machine learning is used — but instead we see an emergent \textit{harmonic synchrony} in a collective, by endowing fairly simple agents with not too complicated update-functions. This is well known in the Multi-Agent Systems \& Swarm Robotics literature \opphoy{\cite{,,}}.}
}